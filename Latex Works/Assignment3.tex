
\date{\today}
\title{\vspace{-1cm}Affine and Metric Correction of Image}


\documentclass[12pt]{article}

\usepackage{graphicx}
\usepackage{mathtools}
\usepackage{cancel}


\author{
  Gupta, Ankesh\\
  \texttt{2015CS10435}
}
% \renewcommand{\labelenumi}{\alph{enumi})}
% \renewcommand{\@seccntformat}[1]{}
% \makeatother
\setcounter{secnumdepth}{0}% disables section numbering

\begin{document}
\maketitle

\section{Problem Statement}
\begin{itemize}
    \item Take image as input and perform \emph{affine correction}.
    \item Take image as input and perform \emph{metric correction}.
\end{itemize}


\section{Implementation}
\begin{enumerate}
    \item For affine correction, 2 sets of \emph{parallel lines} were manually detected.
    \item Each set gives us a corresponding \emph{vanishing point}, giving us the \emph{vanishing line}.
    \item This vanishing line $(l_1,l_2,l_3)$ is then mapped to \emph{line at infinity} using:
    \[
    \begin{bmatrix}
    \centering
    1 & 0 & 0 \\
    0 & 1 & 1 \\
    l_1 & l_2 & l_3
    \end{bmatrix}
    \]
    \item For metric correction, we use the same 4 points as mentioned, 
    \item These 4 points are mapped to an \emph{approximate square}.
\end{enumerate}

\section{Results}
\begin{itemize}
	\item Concatenated \emph{Image,Affine\_Correct,Metric\_Correct} image are present in \emph{CorrectedImage directory}.
\end{itemize}

\end{document}
