
\date{\today}
\title{\vspace{-1cm}Video Stabilization using Warping}


\documentclass[12pt]{article}

\usepackage{graphicx}
\usepackage{mathtools}
\usepackage{cancel}


\author{
  Gupta, Ankesh\\
  \texttt{2015CS10435}
}
% \renewcommand{\labelenumi}{\alph{enumi})}
% \renewcommand{\@seccntformat}[1]{}
% \makeatother
\setcounter{secnumdepth}{0}% disables section numbering

\begin{document}
\maketitle

\section{Problem Statement}
\begin{itemize}
    \item Write the code for the recursive Gauss-Newton process yourself.
    \item Implemente video stabilization using above.
\end{itemize}


\section{Implementation}
\begin{enumerate}
    \item For \emph{image warping}, work of \textbf{\emph{Lucas-Kanade 20 Years On: A Unifying Framework by Simon Baker and Iain Matthews}} was referred. \textbf{Inverse Compositional Image Alignment} was implemented. 
    \item For \emph{video stabilization}, we first ask user for a bounding box/object that would be stabilised. Then all subsequent frames are \emph{warped} to a reference frame(it was the \emph{first frame} in our case).
\end{enumerate}

\section{Tips and Trick}
Since the above was implemented in OpenCV using python, following are some tips and tricks for getting Gauss Newton to converge in the aforementioned language.
\begin{enumerate}
    \item Use the \emph{Inverse Compositional Image Alignment Algorithm} as mentioned above. Simple algorithm mentioned in section 2.2 of paper fails to converge.
    \item Remove the \emph{inverse} in the last step of iterative pseudo code.
    \item Compute \emph{Hessian Inverse} in pre-computation part for speedy iteration.
    \item Avoid \emph{for-loops} and compute all operations using in-built numpy functions for performance(Check reference code).
    \item For video-stabilization, keep an \emph{iteration limit} apart from $\epsilon$, so that some frames that does warp(converge) gets handled.
\end{enumerate}

\section{Assumption}
\begin{enumerate}
    \item The stabilizer corrects motion upto a certain waver.  
    \item The object being stabilized does not move out of scene in subsequent frames.
    \item \emph{Inverse} of Hessian exists.
    \item The transformation is \emph{affine}.    
\end{enumerate}

\section{Results}
% \subsection{Experiment 1}
\begin{table}[h!]
\centering
\begin{tabular}{||c | c ||} 
    \hline
    Epsilon & Iterations \\
    [0.5ex]
    \hline\hline
    0.05 & 1\\
    0.01 & 290\\
    0.005 & 297\\
    0.001 & 315\\
    0.0005 & 581\\
\hline\hline

% 0.05 & 
\end{tabular}
\end{table}

\begin{itemize}
    \item Increasing $\epsilon$ above a certain value, or decreasing below a threshold is futile. \emph{0.005 appeared optimal}.
    \item Larger resolution Images take \emph{longer time} to converge.
    \item Sample warped Images and Videos are shown in respective directories.
    \item The book video has moderate shiver and algorithm successfully tackles it.
    \item The kitty video has large camera motions and algorithms produces and \emph{unstable output}.
\end{itemize}

\section {References}
\begin{enumerate}
    \item Paper mentioned in implementation bullet 1.
    \item Video and Picture Credits:\emph{ Suyash Agarwal, Saket Dingliwal}
\end{enumerate}
\end{document}

